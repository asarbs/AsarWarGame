\chapter{Broń}

\section{Statystyki broni}
\begin{enumerate}
    \item \textbf{Nazwa}
    \item \textbf{Zasięg}
    Bronie ze względu na zasięg dzielimy na:
    \begin{itemize}
        \item \textbf{Do walki wręcz}. Takie bronie można używać tylko jeśli modele stykają się podstawkami. 
        \item \textbf{Drzewcowe}. Takimi broniami możemy zadawać obrażenia jeśli przeciwniki znajduje się 1" od atakującego.   
        \item \textbf{Strzeleckie}. Sa to bronie takie jak dmuchawki, łuki, kusze. Ich zasię dzielimy na Bliski i Daleki. W zasięgu Bliskim strzelający nie otrzymuje żadnych modyfikatorów podczas testu Umiejętności Strzeleckich. Jeśli cel ostrzału znajduje się w zasięgu Dalekim, podczas Testu umiejętności Strzeleckich atakujący otrzymuje modyfikator -3. 
    \end{itemize}
    \item \textbf{Siła}. Modyfikuje ona wartość Zbroi podczas testu 
    \item \textbf{Cechy}. Lista dodatkowych cech jakimi charakteryzuje się dana broń. 
\end{enumerate}

\section{Cechy}
\begin{enumerate}
    \item \textbf{Automatyczna} \index[t]{Cechy Broni - Automatyczna} \label{sec:link_bron_Automatyczna} - Jeśli powiedzie się test trafienia bronią oznaczoną taką cechą nie wykonuje się testu pancerza. Automatycznie zadaje ona obrażenie
    \item \textbf{Bez Celu} \index[t]{Cechy Broni - Bez Celu} \label{sec:link_bron_BezCelu} - Oznacz że celem ataku nie musi być konkretny cel, może nim być dowolnie wybrany punkt na polu walki.
    \item \textbf{Broń rzucana} \index[t]{Cechy Broni - Broń rzucana} \label{sec:link_bron_BronRzucana} - W mechanice gry oznacza to że do testu na trafienie tą bronią wykorzystuje się statystykę Sprawność fizyczna zamiast Umiejętności strzeleckie
    \item \textbf{Wzornik: Duży} \index[t]{Cechy Broni - Wzornik: Duży} \label{sec:link_bron_WzornikDuzy} - Oznacza że po pomyślny zdaniu testu trafienia na celu kładzie się Duży Wzornik a wszystkie modele w pełni objęte tym wzornikiem zostają trafione. 
    \item \textbf{Wzornik: Mały} \index[t]{Cechy Broni - Wzornik: Mały} \label{sec:link_bron_WzornikMaly} - Oznacza że po pomyślny zdaniu testu trafienia na celu kładzie się Mały Wzornik a wszystkie modele w pełni objęte tym wzornikiem zostają trafione. 
    \item \textbf{Ogień} \index[t]{Cechy Broni - Ogień} \label{sec:link_bron_Ogien}
    \item \textbf{Zimno} \index[t]{Cechy Broni - Zimno} \label{sec:link_bron_Zimno}
    \item \textbf{Magiczne} \index[t]{Cechy Broni - Magiczne} \label{sec:link_bron_Magiczne}
    \item \textbf{Trucizny} \index[t]{Cechy Broni - Trucizny} \label{sec:link_bron_Trucizny} W przypadku kiedy model ma więcej niż jedną nie zużytą ranę i Test Pancerze gracz musi wykonać kolejny test tak długo aż nie powiedzie mu się albo nie zostanie wyeliminowany. 
\end{enumerate}

\section{Wzorniki}
\myssection{Mały} \index[t]{Wzorniki Mały} \label{sec:link_bron_Automatyczna} Małe znacznik ma średnicę 3". 
\myssection{Duży} \index[t]{Wzorniki Duży} \label{sec:link_bron_Automatyczna} Duży znacznik ma średnicę 5".
