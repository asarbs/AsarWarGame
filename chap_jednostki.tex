\chapter{Jednostki}

\section{Klasyfikacja jednostek}
\begin{itemize}
    \item \textbf{Model pojedynczy} \index[t]{Model pojedynczy} \label{sec:link_jednostki_model_pojedynczy} - model który może poruszyć się po polu bitwy samodzielnie.
    \item \textbf{Jednostka} \index[t]{Jednostka} \label{sec:link_jednostki_jednostka} - grupa modeli pod dowództwem jednego modelu.
    Dodatkowe zasady wszystkie
    \begin{itemize}
        \item Modele z oddziały muszą znajdować się w promieniu 5" od dowódcy
        \item Wszystkie modele w oddziale wykonują takie same akcje.
    \end{itemize}
    \item \textbf{Kawalerzyści} \index[t]{Kawalerzyści} \label{sec:link_jednostki_Kawalerzysci}  - modele które poruszają się po polu bitwy na grzbiecie wierzchowców. Wierzchowiec i jeździć są integralną jednostką i posiadają jeden zestaw statystyk.
    \item \textbf{Rydwan} \index[t]{Rydwan} \label{sec:link_jednostki_Rydwan} - 
\end{itemize}

\section{Profil jednostki}
Profil składa się z

\begin{itemize}
    \item Nazwy
    \item Klasyfikacji: \textit{Model Pojedynczy / Jednostka / Kawalerzysta / Rydwan}
    \item Statystyki jednostek
    \begin{itemize}
    	\item \textbf{Ruch} \textit{R} \index[t]{Ruch} \label{sec:link_jednostki_R} - ilość cali o którą może poruszyć się jednostka 
    	\item \textbf{Walka wręcz} \textit{WW} \index[t]{Ruch} \label{sec:link_jednostki_WW} - poziom wyszkolenia do walki wręcz
    	\item \textbf{Umiejętności strzeleckie} \textit{US} \index[t]{Umiejętności strzeleckie} \label{sec:link_jednostki_US} - poziom wyszkolenia w walce strzeleckiej
    	\item \textbf{Akcje}  \textit{A} \index[t]{Akcje} \label{sec:link_jednostki_A} - ilość akcji które model może wykonać w jednej turze.
    	\item \textbf{Sprawność fizyczna} \textit{SF} \index[t]{Sprawność fizyczna} \label{sec:link_jednostki_SF} - atrybut stanowiący poziom trudności podczas wszystkich testów fizycznych, stanowi ilość obrażeń jakie zadaje się bronią białą. 
    	\item \textbf{Siła umysłu}  \textit{SU} \index[t]{Siła umysłu} \label{sec:link_jednostki_SU} - atrybut  stanowiący poziom trudności podczas wszystkich testów czarów, przeciwdziałania czarom etc.
    	\item \textbf{Zbroja}  \textit{Z} \index[t]{Zbroja} \label{sec:link_jednostki_Z} - wykorzystywana jest do modyfikowania poziomu trudności testów obrażeń. 
    	\item \textbf{Rany}  \textit{Ra} \index[t]{Rany} \label{sec:link_jednostki_Ra} - ilość obrażeń jakie jednostka może otrzymać zanim zostanie zabita. 
    	\item \textbf{Dostępność}  \textit{D} \index[t]{Dostępność} \label{sec:link_jednostki_D} - mówi nam ile modeli w jednym oddziale możemy wystawić
    	    \begin{itemize}
    	        \item pojedyncza cyfra (np. 1) oznacza że musimy wystawić dokładnie tyle modeli
    	        \item przedział (np. 4-8) oznacza że możemy wystawić nie mniej niż pierwsza cyfra i nie więcej niż druga cyfra modeli. 
    	    \end{itemize}
    	\item \textbf{Wielkość} \textit{W} \index[t]{Wielkość} \label{sec:link_jednostki_W} - Określa wielkość modelu od 1 (Goblin) do 8 (Latający Pożeracz Dusz).
    	\item \textbf{Koszt}  \textit{K} \index[t]{Koszt} \label{sec:link_jednostki_W} - koszt punktowy wystawienia jednego modelu. 
    \end{itemize} 
    \item Lista umiejętności specjalnych
    \item Lista wyposażenia 
\end{itemize}

\myssection{Przykład profilu}

\begin{table}[h]
\caption{Model pojedynczy}

\begin{tabular}{|l|l|l|l|l|l|l|l|l|l|l|l|}
\hline
\multicolumn{1}{c}{Nazwa} & \multicolumn{1}{c}{R} & \multicolumn{1}{c}{WW} & \multicolumn{1}{c}{US} & \multicolumn{1}{c}{A} & \multicolumn{1}{c}{SF} & \multicolumn{1}{c}{SU} & \multicolumn{1}{c}{Z} & \multicolumn{1}{c}{Ra} & \multicolumn{1}{c}{D} & \multicolumn{1}{c}{W} & \multicolumn{1}{c}{K} \\ \hline
Wojownik & 4 & 12 & 11 & 2 & 13 & 13 & 10 & 1 & 1 & 2 & 73 \\ \hline
Umiejętności: \\ \hline
Wyposarzenie: \\ \hline
\end{tabular}

\end{table}

\begin{table}[h]
\caption{Odział}

\begin{tabular}{|l|l|l|l|l|l|l|l|l|l|l|l|}
\hline
\multicolumn{1}{c}{Nazwa} & \multicolumn{1}{c}{R} & \multicolumn{1}{c}{WW} & \multicolumn{1}{c}{US} & \multicolumn{1}{c}{A} & \multicolumn{1}{c}{SF} & \multicolumn{1}{c}{SU} & \multicolumn{1}{c}{Z} & \multicolumn{1}{c}{Ra} & \multicolumn{1}{c}{D} & \multicolumn{1}{c}{W} & \multicolumn{1}{c}{K} \\ \hline
Wojownik & 4 & 12 & 11 & 2 & 13 & 13 & 10 & 1 & 4-8 & 2 & 24  \\ \hline
Dobosz & 4 & 12 & 11 & 2 & 13 & 13 & 10 & 1 & 0-1 & 2 & 27 \\ \hline
Dowóddza & 4 & 12 & 11 & 2 & 13 & 13 & 10 & 1 & 1 & 2 & 26 \\ \hline
Umiejętności: \\ \hline
Wyposarzenie: \\ \hline
\end{tabular}

\end{table}
