\chapter{Jednostki}

\section{Klasyfikacja jednostek}
\begin{itemize}
    \item \textbf{Model pojedynczy} - model który może poruszyć się po polu bitwy samodzielnie.
    \item \textbf{Jednostka} - grupa modeli pod dowództwem jednego modelu.
    Dodatkowe zasady wszystkie
    \begin{itemize}
        \item Modele z oddziały muszą znajdować się w promieniu 5" od dowódcy
        \item Wszystkie modele w oddziale wykonują takie same akcje.
    \end{itemize}
\end{itemize}

\section{Profil jednostki}
Profil składa się z

\begin{itemize}
    \item Nazwy
    \item Statystyki jednostek
    \begin{itemize}
    	\item \textbf{Ruch} - ilość cali o którą może poruszyć się jednostka 
    	\item \textbf{Walka wręcz} - poziom wyszkolenia do walki wręcz
    	\item \textbf{Umiejętności strzeleckie} - poziom wyszkolenia w walce strzeleckiej
    	\item \textbf{Akcje} - ilość akcji które model może wykonać w jednej turze.
    	\item \textbf{Sprawność fizyczna} - atrybut stanowiący poziom trudności podczas wszystkich testów fizycznych, stanowi ilość obrażeń jakie zadaje się bronią białą. 
    	\item \textbf{Siła umysłu} - atrybut  stanowiący poziom trudności podczas wszystkich testów czarów, przeciwdziałania czarom etc.
    	\item \textbf{Zbroja} - wykorzystywana jest do modyfikowania poziomu trudności testów obrażeń. 
    	\item \textbf{Rany} - ilość obrażeń jakie jednostka może otrzymać zanim zostanie zabita. 
    	\item \textbf{Koszt} - koszt punktowy wystawienia jednostki. 
    \end{itemize} 
    \item Lista umiejętności specjalnych
    \item Lista wyposażenia 
\end{itemize}

\subsection{Przykład profilu}
\subsubsection{Model pojedynczy}

\subsubsection{Odział}



