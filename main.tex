\documentclass[a4paper,12pt,twoside]{amsbook}


%\usepackage[a4paper, left=1.0cm, right=1.0cm, top=1.0cm, bottom=2.0cm, headsep=1.0cm]{geometry}
\usepackage[a4paper]{geometry}
\usepackage[utf8]{inputenc}
\usepackage[MeX]{polski}
\usepackage[pdftex]{graphicx}
\usepackage{amsmath}
\usepackage{color}
\usepackage{fancyhdr}
\usepackage{float}
\usepackage{hyperref}
\usepackage{siunitx}
\usepackage{imakeidx}
\makeindex[name=t,title=Indeks terminów,columns=3]


\numberwithin{figure}{chapter}
\numberwithin{table}{chapter}

\pagestyle{fancy}

\usepackage[T1]{fontenc}
\usepackage[light,condensed,math]{anttor}

\definecolor{orange}{rgb}{1,0.5,0}
\definecolor{red}{rgb}{1,0,0}
\definecolor{green}{rgb}{0,0.5,0}
\definecolor{bule}{rgb}{0,0,1}

\newcommand{\myssection}[1] {
\subsection{#1}
\hfill\\
}

\begin{document}
\begin{center}
\begin{minipage}{0.75\linewidth}
    \centering
    %\includegraphics[width=0.3\linewidth]{logo.pdf}
    % \rule{0.4\linewidth}{0.15\linewidth}\par
    \vspace{3cm}
%title
    {\uppercase{\Large GRW v2.0\par}}
    \vspace{6cm}
%Author's name
    Bartosz "Asar" Skorupa - bartosz@skorupa.net
    \vspace{3cm}

%Date
    {\Large \today}
\end{minipage}
\end{center}
\tableofcontents

\listoffigures
\listoftables

\chapter{Zasady ogóle}

\section{Przebieg gry}

\subsection{Rozpoczęcie gry}

\subsubsection{Wybór sił}

\subsubsection{Przygotowanie pola gry}

\subsubsection{Rozstawienie}

\subsection{Tura}

\subsubsection{Inicjatywa}

\subsubsection{Wykonanie akcji}

\subsection{Warunki zwycięstwa}

\subsubsection{Scenariusze}

\section{Pomiar odległości}
W trakcie gry pomiarów odległości takich jak ruch, zasięgi broni dokonuje się miarką calową. Gracze mogę dokonywać pomiarów odległości w każdym momencie w trackie gry. 

\section{Pole widzenia}

\subsection{Pieszy}
Pole widzenia pieszego to \ang{180} oznaczone przez gracza na podstawce, chyba że zasady specjalne modelu mówią inaczej

\subsection{Kawaleria}
Domyślne pole widzenia kawalerzysty to \ang{360} oznaczone przez gracza na podstawce. 

\subsection{Zasłanianie modeli przez inne modele}
Modele o takiej samej lub większej wielkości statystyki zasłaniają model za sobą w linie widzenia. 

\section{Testy}

Wszystkie rzuty w grze wykonuje się przy pomocy kostek dwudziestościennych (k20).

\subsection{Testy atrybutów} wykonuje się po przez rzut k20, wyniki rzutu musi być mniejszy bądź równy wartości zmodyfikowanego atrybutu aby test został uznany za zdany. 

\subsection{Testy przeciwstawne} wykonuje się po przez rzut k20, wynik rzutu musi być mniejszy bądź równy wartości zmodyfikowanego atrybutu który testujemy i większy od rzutu przeciwnika. W przypadku wyrzucenia równego wyniku na kostkach, gracze porównują wartość zmodyfikowanego atrybutu. Gracz z większym zmodyfikowanym atrybutem wygrywa Test przeciwstawny. 
\subsection{Testy pancerza} wykonuje się po przez rzut k20 na atrybut Zbroja zmodyfikowany przez siłę broni. Jeśli wynik rzutu będzie mniejszy bądź równy zmodyfikowanej wartości atrybutu Zbroi gracz zdał test i model \underline{nie} otrzymuje obrażeń.


\section{Poruszanie się}
\subsection{Zasady ogólne}
Atrybut Ruch oznacza maksymalna liczbę cali o które może przesunąć się model. Na końcu ruchu model może być zwrócony w dowolonym kierunku. Model może przekroczyć dowolny element terenu który jest o połowę niższy od niego oraz jego szerokość jest o połowe mniejsza od średnicy podstawki bez ponoszenia dodatkowego kosztu. W każdym innym przypadku model musi wykonać akcję Skok. 

\begin{table}[h]
\caption{Trudny teren}
\begin{tabular}{|l|l|l|l|}
\hline
\multicolumn{1}{c}{Type jednostki} & \multicolumn{1}{c}{Zwykły} & \multicolumn{1}{c}{Trudny} & \multicolumn{1}{c}{Nie możliwy do przebycia} \\ \hline
Piechota & pełnym ruchem & połową & nie może poruszać się \\ \hline
Kawaleria & pełnym ruchem & nie może poruszać się & nie może poruszać się \\ \hline
\end{tabular}

\end{table}

\section{Zasięg dowodzenia}
\subsection{Model pojedynczy}

\subsection{Jednostka z dowódcą}

\subsection{Jednostka bez dowódcy}


\section{Walka}
\subsection{Walka wręcz}
Jako walczące wręcz uznajemy wrogie modele jeśli ich podstawki się stykają. Walka odbywa się w sekwencji.
\begin{enumerate}
    \item Atakujący model wykonuje szarżę
    \item Model broniący się deklaruje akcje reakcji: Atak, Unik, Rzucenie czaru, Oderwanie się od przeciwnika
    \item Gracze wykonują Test przeciwstawnych zadeklarowanych akcji.
    \item Gracze rozwiązują akcje zgodnie z wynikiem Testu przeciwstawnego.
    \item Gracz który przegrał wykonuje Test pancerza.
\end{enumerate}

\subsubsection{Kilku atakujących} jednego przeciwnika zyskuje znaczącą przewagę. W trakcie wykonywania testu Walki wręcz model uzyskuje modyfikator \textbf{+1} do swojej statystyki WW za każdego kompana uczestniczącego w tej walce. 

\subsubsection{Ilu przeciwników może atakować jeden model?} Model może być atakowany w walce wręcz przez tyle modeli ile może dość do kontaktu podstawek. 

\subsection{Walka dystansowa}
Walka strzelecka odbywa się w sekwencji
\begin{enumerate}
    \item Atakujący model deklaruje akcję Strzał
    \item Model broniący się deklaruje akcje reakcji: Strzał, Unik, Rzucenie czaru. 
    \item Gracze wykonują Test przeciwstawnych zadeklarowanych akcji.
    \item Gracze rozwiązują akcje zgodnie z wynikiem Testu przeciwstawnego. 
    \item Gracz który przegrał wykonuje Test pancerza.
\end{enumerate}

\subsubsection{Osłona}
Model jest za osłoną jeśli syka się z elementem terenu oraz przynajmniej 20\% modelu jest zakryte przez ten element terenu. Model strzelający do modelu za osłoną otrzymuje modyfikator \textbf{-3} to testu US oraz \textbf{+3} do Testu pancerza

\subsubsection{Strzelanie do walczących wręcz}
Strzelanie do walczących wręcz obarczone jest dodatkowym modyfikatorem \textbf{-6}. Jeśli wyniki rzutu będzie mniejszy niż test Umiejętności strzeleckich -6 to został trafiony wrogi model. Jeśli wyniki będzie z przedziału wartość zmodyfikowanych US - 6 a zmodyfikowanych US został trafiony przyjazny model. 

\textbf{Przykład}
\chapter{Jednostki}

\section{Klasyfikacja jednostek}
\begin{itemize}
    \item \textbf{Model pojedynczy} - model który może poruszyć się po polu bitwy samodzielnie.
    \item \textbf{Jednostka} - grupa modeli pod dowództwem jednego modelu.
    Dodatkowe zasady wszystkie
    \begin{itemize}
        \item Modele z oddziały muszą znajdować się w promieniu 5" od dowódcy
        \item Wszystkie modele w oddziale wykonują takie same akcje.
    \end{itemize}
    \item \textbf{Kawalerzyści} - modele które poruszają się po polu bitwy na grzbiecie wierzchowców. Wierzchowiec i jeździeć są integralną jednostką i posiadają jeden zestaw statystyk.
    \item \textbf{Rydwan} - 
\end{itemize}

\section{Profil jednostki}
Profil składa się z

\begin{itemize}
    \item Nazwy
    \item Klasyfikacji: \textit{Model Pojedynczy / Jednostka / Kawalerzysta / Rydwan}
    \item Statystyki jednostek
    \begin{itemize}
    	\item \textbf{Ruch} \textit{R} - ilość cali o którą może poruszyć się jednostka 
    	\item \textbf{Walka wręcz}  \textit{WW} - poziom wyszkolenia do walki wręcz
    	\item \textbf{Umiejętności strzeleckie}  \textit{US} - poziom wyszkolenia w walce strzeleckiej
    	\item \textbf{Akcje}  \textit{A} - ilość akcji które model może wykonać w jednej turze.
    	\item \textbf{Sprawność fizyczna}  \textit{SF} - atrybut stanowiący poziom trudności podczas wszystkich testów fizycznych, stanowi ilość obrażeń jakie zadaje się bronią białą. 
    	\item \textbf{Siła umysłu}  \textit{SU} - atrybut  stanowiący poziom trudności podczas wszystkich testów czarów, przeciwdziałania czarom etc.
    	\item \textbf{Zbroja}  \textit{Z} - wykorzystywana jest do modyfikowania poziomu trudności testów obrażeń. 
    	\item \textbf{Rany}  \textit{Ra} - ilość obrażeń jakie jednostka może otrzymać zanim zostanie zabita. 
    	\item \textbf{Dostępność}  \textit{D} - mówi nam ile modeli w jednym oddziale możemy wystawić
    	    \begin{itemize}
    	        \item pojedyncza cyfra (np. 1) oznacza że musimy wystawić dokładnie tyle modeli
    	        \item przedział (np. 4-8) oznacza że możemy wystawić nie mniej niż pierwsza cyfra i nie więcej niż druga cyfra modeli. 
    	    \end{itemize}
    	\item \textbf{Wielkość} \textit{W} - Określa wielkość modelu od 1 (Goblin) do 8 (Latający Pożeracz Dusz).
    	\item \textbf{Koszt}  \textit{K} - koszt punktowy wystawienia jednego modelu. 
    \end{itemize} 
    \item Lista umiejętności specjalnych
    \item Lista wyposażenia 
\end{itemize}

\myssection{Przykład profilu}


\begin{table}[h]
\caption{Model pojedynczy}

\begin{tabular}{|l|l|l|l|l|l|l|l|l|l|l|l|}
\hline
\multicolumn{1}{c}{Nazwa} & \multicolumn{1}{c}{R} & \multicolumn{1}{c}{WW} & \multicolumn{1}{c}{US} & \multicolumn{1}{c}{A} & \multicolumn{1}{c}{SF} & \multicolumn{1}{c}{SU} & \multicolumn{1}{c}{Z} & \multicolumn{1}{c}{Ra} & \multicolumn{1}{c}{D} & \multicolumn{1}{c}{W} & \multicolumn{1}{c}{K} \\ \hline
Wojownik & 4 & 12 & 11 & 2 & 13 & 13 & 10 & 1 & 1 & 2 & 73 \\ \hline
Umiejętności: \\ \hline
Wyposarzenie: \\ \hline
\end{tabular}

\end{table}

\begin{table}[h]
\caption{Odział}

\begin{tabular}{|l|l|l|l|l|l|l|l|l|l|l|l|}
\hline
\multicolumn{1}{c}{Nazwa} & \multicolumn{1}{c}{R} & \multicolumn{1}{c}{WW} & \multicolumn{1}{c}{US} & \multicolumn{1}{c}{A} & \multicolumn{1}{c}{SF} & \multicolumn{1}{c}{SU} & \multicolumn{1}{c}{Z} & \multicolumn{1}{c}{Ra} & \multicolumn{1}{c}{D} & \multicolumn{1}{c}{W} & \multicolumn{1}{c}{K} \\ \hline
Wojownik & 4 & 12 & 11 & 2 & 13 & 13 & 10 & 1 & 4-8 & 2 & 24  \\ \hline
Dobosz & 4 & 12 & 11 & 2 & 13 & 13 & 10 & 1 & 0-1 & 2 & 27 \\ \hline
Dowóddza & 4 & 12 & 11 & 2 & 13 & 13 & 10 & 1 & 1 & 2 & 26 \\ \hline
Umiejętności: \\ \hline
Wyposarzenie: \\ \hline
\end{tabular}

\end{table}

\chapter{Broń}

\section{Statystyki broni}
\begin{enumerate}
    \item \textbf{Nazwa}
    \item \textbf{Zasięg}
    Bronie ze względu na zasięg dzielimy na:
    \begin{itemize}
        \item \textbf{Do walki wręcz}. Takie bronie można używać tylko jeśli modele stykają się podstawkami. 
        \item \textbf{Drzewcowe}. Takimi broniami możemy zadawać obrażenia jeśli przeciwniki znajduje się 1" od atakującego.   
        \item \textbf{Strzeleckie}. Sa to bronie takie jak dmuchawki, łuki, kusze. Ich zasię dzielimy na Bliski i Daleki. W zasięgu Bliskim strzelający nie otrzymuje żadnych modyfikatorów podczas testu Umiejętności Strzeleckich. Jeśli cel ostrzału znajduje się w zasięgu Dalekim, podczas Testu umiejętności Strzeleckich atakujący otrzymuje modyfikator -3. 
    \end{itemize}
    \item \textbf{Siła}. Modyfikuje ona wartość Zbroi podczas testu 
    \item \textbf{Cechy}. Lista dodatkowych cech jakimi charakteryzuje się dana broń. 
\end{enumerate}

\section{Cechy}
\begin{enumerate}
    \item \textbf{Automatyczna} - Jeśli powiedzie się test trafienia bronią oznaczoną taką cechą nie wykonuje się testu pancerza. Automatycznie zadaje ona obrażenie
    \item \textbf{Bez Celu} - Oznacz że celem ataku nie musi być konkretny cel, może nim być dowolnie wybrany punkt na polu walki.
    \item \textbf{Broń rzucana} - W mechanice gry oznacza to że do testu na trafienie tą bronią wykorzystuje się statystykę Sprawność fizyczna zamiast Umiejętności strzeleckie
    \item \textbf{Wzornik: Duży} - Oznacza że po pomyślny zdaniu testu trafienia na celu kładzie się Duży Wzornik a wszystkie modele w pełni objęte tym wzornikiem zostają trafione. 
    \item \textbf{Wzornik: Mały} - Oznacza że po pomyślny zdaniu testu trafienia na celu kładzie się Mały Wzornik a wszystkie modele w pełni objęte tym wzornikiem zostają trafione. 
    \item \textbf{Ogień}
    \item \textbf{Zimno}
    \item \textbf{Magiczne}
    \item \textbf{Trucizny} W przypadku kiedy model ma więcej niż jedną nie zurzytą ranę i Test Pancerze gracz musi wykonać kolejny test tak długo aż nie powiedzie mu się albo nie zostanie wyeliminowany. 
\end{enumerate}

\section{Wzorniki}
\myssection{Mały} Małe znacznik ma średnicę 3". 
\myssection{Duży} Duży znacznik ma średnicę 5".

\chapter{Akcje}

\section{Atak}

\section{Strzał}

\section{Unik}

\section{Oderwanie się od przeciwnika}

\section{Rzucenie czaru}

\section{Skok}

\section{Czekanie}

\section{Wydawanie rozkazów}


\chapter{Umiejętności specjalne}

\section{Krycie się} 
\index[t]{Krycie się} \label{sec:link_uw_krycie_sie}

\section{Skradanie się} 
\index[t]{Skradanie się} \label{sec:link_uw_skradnie_sie}

\section{Odporność na ...} 
\index[t]{Odporność na ...} \label{sec:link_uw_odpornosc}

\section{Ściana tarcz} 
\index[t]{Ściana tarcz} \label{sec:link_uw_sciana_tarcz}

\section{Finta} \index[t]{Finta} \label{sec:link_uw_finta}

\section{Staranowanie} \index[t]{Staranowanie} \label{sec:link_uw_staranowanie}

\section{Zabójczy cios} \index[t]{Zabójczy cios} \label{sec:link_uw_zabojczy_cios}

\section{Sokole oko} \index[t]{Sokole oko} \label{sec:link_uw_sokole_oko}

\section{Grad strzał} \index[t]{Grad strzał} \label{sec:link_uw_grad_strzal}

\section{Berserker} \index[t]{Berserker} \label{sec:link_uw_berserker}

\section{Atak grupowy} \index[t]{Atak grupowy} \label{sec:link_uw_atak_grupowe}

\section{Jeż} \index[t]{Jeż} \label{sec:link_uw_jez}

\section{Dowódca} \index[t]{Dowódca} \label{sec:link_uw_dowdoca}

\section{Generał} \index[t]{Generał} \label{sec:link_uw_general}

\section{Infiltracja} \index[t]{Infiltracja} \label{sec:link_uw_inflirtacja}

\chapter{Wyposażenie} 

\chapter{Magia} 

\section{Czary}

\myssection{Cechy czarów} 




\chapter{Scenariusze}

\section{Przełamanie obrony}
\label{sec:link_scenariusze_przelamanie_obrony} \index[t]{Przełamanie obrony}
\section{Anihilacja}
\label{sec:link_scenariusze_anihilacja} \index[t]{Anihilacja}
\section{Ostatni bastion}
\label{sec:link_scenariusze_ostatni_bastion} \index[t]{Ostatni bastion}
\section{Zdobyć artefakt}
\label{sec:link_scenariusze_zdobyc_artefakt} \index[t]{Zdobyć artefakt}

\chapter{Budowanie punktacji w grze}
Ten rozdział zawiera informacje na jakich zasadach należy budować armii. 

\section{Jednostki}

Domyślne statystyki dla każdej jednostki w grze to 

\begin{table}[h]
\caption{Podstawowe statystyki jednostki}

    \begin{tabular}{|l|l|l|l|l|l|l|l|l|l|}
    \hline
    \multicolumn{1}{c}{R} & \multicolumn{1}{c}{WW} & \multicolumn{1}{c}{US} & \multicolumn{1}{c}{A} & \multicolumn{1}{c}{SF} & \multicolumn{1}{c}{SU} & \multicolumn{1}{c}{Z} & \multicolumn{1}{c}{Ra} & \multicolumn{1}{c}{D} & \multicolumn{1}{c}{K} \\ \hline
    4 & 10 & 10 & 2 & 10 & 10 & 10 & 1 & - & Wyliczany z tych zasad \\ \hline
    \end{tabular}

\end{table}

Taki zestaw statystyk jest darmowy, kosztuje on 16 punktów. 
Zmiana domyślnych statystyk kosztuje zgodnie z tabelami. 
\begin{table}[h]
\caption{Kosz zmiany statystyk: WW, US, SF, SU i Z}
\resizebox{\textwidth}{!}{
    \begin{tabular}{|l|l|l|l|l|l|l|l|l|l|l|l|l|l|l|l|l|l|l|l|l|l|l|l|l|l|}
    \hline
    Poziom & 1   & 2   & 3   & 4   & 5   & 6   & 7  & 8  & 9  & 10 & 11 & 12 & 13 & 14 & 15 & 16 & 17 & 18 & 19 & 20 & 21 & 22 & 23 & 24 & 25 \\ \hline
    Koszt  & -20 & -18 & -16 & -14 & -12 & -10 & -8 & -6 & -4 & 2  & 4  & 6  & 8  & 10 & 12 & 14 & 16 & 18 & 20 & 22 & 24 & 26 & 28 & 30 & 33 \\ \hline
    \end{tabular}
    }
\end{table}

\begin{table}[h]
\caption{Kosz zmiany statystyk: A}
    \begin{tabular}{|l|l|l|l|l|l|l|l|l|l|l|l|l|l|l|l|l|l|l|l|l|l|l|l|l|l|}
    \hline
    Poziom & 0 & 1 & 2 & 3 & 4 & 5 & 6 & 7 & 8 & 9 \\ \hline
    Koszt  & -6 & -2 & 2 & 6 & 10 & 14 & 18 & 22 & 26 & 30 \\ \hline
    \end{tabular}
\end{table}

\begin{table}[h]
\caption{Kosz zmiany statystyk: Ra}
    \begin{tabular}{|l|l|l|l|l|l|l|l|l|l|l|l|l|l|l|l|l|l|l|l|l|l|l|l|l|l|}
    \hline
    Poziom & 0 & 1 & 2 & 3 & 4 & 5 & 6 & 7 & 8 & 9 \\ \hline
    Koszt  & -4 & -1 & 2 & 5 & 8 & 11 & 14 & 17 & 20 & 23 \\ \hline
    \end{tabular}
\end{table}

Minus przed kosztem obniża koszt jednostki. To znaczy  jeśli chcemy zmienić jednostce statystykę WW z 10 na 9 koszt modelu zmaleje o 4. 
Na koszt jednostki wpływa wybrane uzbrojenie, wyposażenie i umiejętności specjalne. Kosz podany przy każdtym z tych elementów dodaje się do kosztu jednostki. 

\section{Broń}

Koszt broni składa się z kosztu jej Siły, Zasięgu oraz kosztu wybranych cech. 

\myssection{Koszt Siły}

Domyślną siłą broni jest 2 a jej koszt to 2.
\begin{table}[h]
\caption{Kosz zmiany statystyk Siła broni}
    \begin{tabular}{|l|l|l|l|l|l|l|l|l|l|l|l|l|l|l|l|l|l|l|l|l|l|l|l|l|l|}
    \hline
    Poziom & 0 & 1 & 2 & 3 & 4 & 5 & 6 & 7 & 8 & 9 \\ \hline
    Koszt  & -6 & -2 & 2 & 6 & 10 & 14 & 18 & 22 & 26 & 30 \\ \hline
    \end{tabular}
\end{table}

Podobnie jak w przypadku jednostek 

\myssection{Koszt Zasięgu}
Domyślnym zasięgiem broni jest Walka Wręcz i jest on darmowa. Aby broń stała się drzewcowa należy jej koszt podnieść o 2. 
Jeśli mówimy o broniach zasięgowych do domyślnymi, darmowymi zasięgami są: 
\begin{itemize}
    \item Zasięg Bliski: 12"
    \item Zasięg Daleki: 16". Zasięg daleki musi być co najmniej o 4" większy od zasięgu bliskiego. 
\end{itemize}

\begin{table}[h]
\caption{Kosz zmiany Zasięgu Bliskiego}
\resizebox{\textwidth}{!}{
    \begin{tabular}{|l|l|l|l|l|l|l|l|l|l|l|l|l|l|l|l|l|l|l|l|l|l|l|l|l|l|}
    \hline
        Zasięg & 1 & 2 & 3 & 4 & 5 & 6 & 7 & 8 & 9 & 10 & 11 & 12 & 13 & 14 & 15 & 16 & 17 & 18 & 19 & 20 & 21 & 22 & 23 \\ \hline
        Koszt & -11 & -10 & -9 & -8 & -7 & -6 & -5 & -4 & -3 & -2 & -1 & 0 & 1 & 2 & 3 & 4 & 5 & 6 & 7 & 8 & 9 & 10 & 11 \\ \hline
    \end{tabular}
    }
\end{table}

% \begin{table}[h]
% \caption{Kosz zmiany Zasięgu Dalekiego}
% \resizebox{\textwidth}{!}{
%     \begin{tabular}{|l|l|l|l|l|l|l|l|l|l|l|l|l|l|l|l|l|l|l|l|l|l|l|l|l|l|}
%     \hline
% Zasięg & 1 & 2 & 3 & 4 & 5 & 6 & 7 & 8 & 9 & 10 & 11 & 12 & 13 & 14 & 15 & 16 & 17 & 18 & 19 & 20 & 21 & 22 & 23 & 24 & 25 & 26 & 27 & 28 & 29 & 30 & 31 & 32  \\ \hline
% Koszt & -15 & -14 & -13 & -12 & -11 & -10 & -9 & -8 & -7 & -6 & -5 & -4 & -3 & -2 & -1 & 0 & 1 & 2 & 3 & 4 & 5 & 6 & 7 & 8 & 9 & 10 & 11 & 12 & 13 & 14 & 15 & 16 \\ \hline
%     \end{tabular}
%     }
% \end{table}

\begin{table}[h]
\caption{Kosz zmiany Zasięgu Dalekiego}
    \begin{tabular}{|l|l|}
    \hline
    Zasięg & Koszt \\ \hline
    1 & -15  \\ 
    2 & -14 \\ 
    3 & -13 \\ 
    4 & -12 \\ 
    5 & -11 \\ 
    6 & -10 \\ 
    7 & -9 \\ 
    8 & -8 \\ 
    9 & -7 \\ 
    10 & -6 \\ 
    11 & -5 \\ 
    12 & -4 \\ 
    13 & -3 \\ 
    14 & -2 \\ 
    15 & -1 \\
    16 & 0 \\ 
    17 & 1 \\ 
    18 & 2 \\ 
    19 & 3 \\ 
    20 & 4 \\ 
    21 & 5 \\ 
    22 & 6 \\ 
    23 & 7 \\ 
    24 & 8 \\ 
    25 & 9 \\ 
    26 & 10 \\ 
    27 & 11 \\ 
    28 & 12 \\ 
    29 & 13 \\ 
    30 & 14 \\ 
    31 & 15 \\ 
    22 & 16 \\ 
    \end{tabular}
\end{table}

\myssection{Koszt Cechy Broni}

Kosz cechy broni wybieralny jest arbitralnie przez autora. 

\section{Umiejętności specjalne}

\section{Wyposażenie}

\section{Czary}
%\input{biblio.tex}

\printindex[t]

\end{document}