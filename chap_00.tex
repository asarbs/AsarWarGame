\chapter{Zasady ogóle}

\section{Pomiar odległości}
W trakcie gry pomiarów odległości takich jak ruch, zasięgi broni dokonuje się miarką calową. Gracze mogę dokonywać pomiarów odległości w każdym momencie w trackie gry. 

\section{Testy}

Wszystkie rzuty w grze wykonuje się przy pomocy kostek dwudziestościennych (k20).

\subsection{Testy atrybutów} wykonuje się po przez rzut k20, wyniki rzutu musi być mniejszy bądź równy wartości zmodyfikowanego atrybutu aby test został uznany za zdany. 

\subsection{Testy przeciwstawne} wykonuje się po przez rzut k20, wynik rzutu musi być mniejszy bądź równy wartości zmodyfikowanego atrybutu który testujemy i większy od rzutu przeciwnika. W przypadku wyrzucenia równego wyniku na kostkach, gracze porównują wartość zmodyfikowanego atrybutu. Gracz z większym zmodyfikowanym atrybutem wygrywa Test przeciwstawny. 
\subsection{Testy pancerza} wykonuje się po przez rzut k20 na atrybut Zbroja zmodyfikowany przez siłę broni. Jeśli wynik rzutu będzie mniejszy bądź równy zmodyfikowanej wartości atrybutu Zbroi gracz zdał test i model \underline{nie} otrzymuje obrażeń.


\section{Poruszanie się}
\subsection{Zasady ogólne}
Atrybut Ruch oznacza maksymalna liczbę cali o które może przesunąć się model. Na końcu ruchu model może być zwrócony w dowolonym kierunku. Model może przekroczyć dowolny element terenu który jest o połowę niższy od niego oraz jego szerokość jest o połowe mniejsza od średnicy podstawki bez ponoszenia dodatkowego kosztu. W każdym innym przypadku model musi wykonać akcję Skok. 

\begin{table}[h]
\caption{Trudny teren}
\begin{tabular}{|l|l|l|l|}
\hline
\multicolumn{1}{c}{Type jednostki} & \multicolumn{1}{c}{Zwykły} & \multicolumn{1}{c}{Trudny} & \multicolumn{1}{c}{Nie możliwy do przebycia} \\ \hline
Piechota & pełnym ruchem & połową & nie może poruszać się \\ \hline
Kawaleria & pełnym ruchem & nie może poruszać się & nie może poruszać się \\ \hline
\end{tabular}

\end{table}


\section{Walka}
\subsection{Walka wręcz}
Jako walczące wręcz uznajemy wrogie modele jeśli ich podstawki się stykają. Walka odbywa się w sekwencji.
\begin{enumerate}
    \item Atakujący model wykonuje szarżę
    \item Model broniący się deklaruje akcje reakcji: Atak, Unik, Rzucenie czaru. 
    \item Gracze wykonują Test przeciwstawnych zadeklarowanych akcji.
    \item Gracze rozwiązują akcje zgodnie z wynikiem Testu przeciwstawego. 
\end{enumerate}
\subsection{Walka dystansowa}
Walka strzelecka odbywa się w sekwencji
\begin{enumerate}
    \item Atakujący model deklaruje akcję Strzał
    \item Model broniący się deklaruje akcje reakcji: Strzał, Unik, Rzucenie czaru. 
    \item Gracze wykonują Test przeciwstawnych zadeklarowanych akcji.
    \item Gracze rozwiązują akcje zgodnie z wynikiem Testu przeciwstawego. 
\end{enumerate}

