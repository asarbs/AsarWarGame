\chapter{Zasady ogóle}

\section{Testy}

Wszystkie rzuty w grze wykonuje się przy pomocy kostek dwudziestościennych (k20).

\subsection{Testy atrybutów} wykonuje się po przez rzut k20, wyniki rzutu musi być mniejszy bądź równy wartości zmodyfikowanego atrybutu aby test został uznany za zdany. 

\subsection{Testy przeciwstawne} wykonuje się po przez rzut k20, wynik rzutu musi być mniejszy bądź równy wartości zmodyfikowanego atrybutu który testujemy i większy od rzutu przeciwnika. W przypadku wyrzucenia równego wyniku na kostkach, gracze porównują wartość zmodyfikowanego atrybutu. Gracz z większym zmodyfikowanym atrybutem wygrywa Test przeciwstawny. 
\subsection{Testy pancerza} wykonuje się po przez rzut k20 na atrybut Zbroja zmodyfikowany przez siłę broni. Jeśli wynik rzutu będzie mniejszy bądź równy zmodyfikowanej wartości atrybutu Zbroi gracz zdał test i model \underline{nie} otrzymuje obrażeń.


\section{Poruszanie się}
\subsection{Zasady ogólne}
Atrybut Ruch oznacza maksymalna liczbę cali o które może przesunąć się model. Na końcu ruchu model może być zwrócony w dowolonym kierunku. Model może przekroczyć dowolny element terenu który jest o połowę niższy od niego oraz jego szerokość jest o połowe mniejsza od średnicy podstawki bez ponoszenia dodatkowego kosztu. W każdym innym przypadku model musi wykonać akcję Skok. 
\subsection{Trudny teren}

\textcolor{red}{tutaj należy wstawić tabelkę po tym jak okeślimy jakie typy jednostek będą w grze} 
\section{Walka}
\subsection{Walka wręcz}
Jako walczące wręcz uznajemy wrogie modele jeśli ich podstawki się stykają. Walka odbywa się w sekwencji.
\begin{enumerate}
    \item Atakujący model wykonuje szarżę
    \item Model broniący się deklaruje akcje reakcji: Atak, Unik, Rzucenie czaru. 
    \item Gracze wykonują Test przeciwstawnych zadeklarowanych akcji.
    \item Gracze rozwiązują akcje zgodnie z wynkiem Testu przeciwstawego. 
\end{enumerate}
\subsection{Walka dystansowa}




