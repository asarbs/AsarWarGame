\chapter{Budowanie punktacji w grze}
Ten rozdział zawiera informacje na jakich zasadach należy budować armii. 

\section{Jednostki}

Domyślne statystyki dla każdej jednostki w grze to 

\begin{table}[h]
\caption{Podstawowe statystyki jednostki}

    \begin{tabular}{|l|l|l|l|l|l|l|l|l|l|}
    \hline
    \multicolumn{1}{c}{R} & \multicolumn{1}{c}{WW} & \multicolumn{1}{c}{US} & \multicolumn{1}{c}{A} & \multicolumn{1}{c}{SF} & \multicolumn{1}{c}{SU} & \multicolumn{1}{c}{Z} & \multicolumn{1}{c}{Ra} & \multicolumn{1}{c}{D} & \multicolumn{1}{c}{K} \\ \hline
    4 & 10 & 10 & 2 & 10 & 10 & 10 & 1 & - & Wyliczany z tych zasad \\ \hline
    \end{tabular}

\end{table}

Taki zestaw statystyk jest darmowy, kosztuje on 16 punktów. 
Zmiana domyślnych statystyk kosztuje zgodnie z tabelami. 
\begin{table}[h]
\caption{Kosz zmiany statystyk: WW, US, SF, SU i Z}
\resizebox{\textwidth}{!}{
    \begin{tabular}{|l|l|l|l|l|l|l|l|l|l|l|l|l|l|l|l|l|l|l|l|l|l|l|l|l|l|}
    \hline
    Poziom & 1   & 2   & 3   & 4   & 5   & 6   & 7  & 8  & 9  & 10 & 11 & 12 & 13 & 14 & 15 & 16 & 17 & 18 & 19 & 20 & 21 & 22 & 23 & 24 & 25 \\ \hline
    Koszt  & -20 & -18 & -16 & -14 & -12 & -10 & -8 & -6 & -4 & 2  & 4  & 6  & 8  & 10 & 12 & 14 & 16 & 18 & 20 & 22 & 24 & 26 & 28 & 30 & 33 \\ \hline
    \end{tabular}
    }
\end{table}

\begin{table}[h]
\caption{Kosz zmiany statystyk: A}
    \begin{tabular}{|l|l|l|l|l|l|l|l|l|l|l|l|l|l|l|l|l|l|l|l|l|l|l|l|l|l|}
    \hline
    Poziom & 0 & 1 & 2 & 3 & 4 & 5 & 6 & 7 & 8 & 9 \\ \hline
    Koszt  & -6 & -2 & 2 & 6 & 10 & 14 & 18 & 22 & 26 & 30 \\ \hline
    \end{tabular}
\end{table}

\begin{table}[h]
\caption{Kosz zmiany statystyk: Ra}
    \begin{tabular}{|l|l|l|l|l|l|l|l|l|l|l|l|l|l|l|l|l|l|l|l|l|l|l|l|l|l|}
    \hline
    Poziom & 0 & 1 & 2 & 3 & 4 & 5 & 6 & 7 & 8 & 9 \\ \hline
    Koszt  & -4 & -1 & 2 & 5 & 8 & 11 & 14 & 17 & 20 & 23 \\ \hline
    \end{tabular}
\end{table}

Minus przed kosztem obniża koszt jednostki. To znaczy  jeśli chcemy zmienić jednostce statystykę WW z 10 na 9 koszt modelu zmaleje o 4. 
Na koszt jednostki wpływa wybrane uzbrojenie, wyposażenie i umiejętności specjalne. Kosz podany przy każdtym z tych elementów dodaje się do kosztu jednostki. 

\section{Broń}

Koszt broni składa się z kosztu jej Siły, Zasięgu oraz kosztu wybranych cech. 

\myssection{Koszt Siły}

Domyślną siłą broni jest 2 a jej koszt to 2.
\begin{table}[h]
\caption{Kosz zmiany statystyk Siła broni}
    \begin{tabular}{|l|l|l|l|l|l|l|l|l|l|l|l|l|l|l|l|l|l|l|l|l|l|l|l|l|l|}
    \hline
    Poziom & 0 & 1 & 2 & 3 & 4 & 5 & 6 & 7 & 8 & 9 \\ \hline
    Koszt  & -6 & -2 & 2 & 6 & 10 & 14 & 18 & 22 & 26 & 30 \\ \hline
    \end{tabular}
\end{table}

Podobnie jak w przypadku jednostek 

\myssection{Koszt Zasięgu}
Domyślnym zasięgiem broni jest Walka Wręcz i jest on darmowa. Aby broń stała się drzewcowa należy jej koszt podnieść o 2. 
Jeśli mówimy o broniach zasięgowych do domyślnymi, darmowymi zasięgami są: 
\begin{itemize}
    \item Zasięg Bliski: 12"
    \item Zasięg Daleki: 16". Zasięg daleki musi być co najmniej o 4" większy od zasięgu bliskiego. 
\end{itemize}

\begin{table}[h]
\caption{Kosz zmiany Zasięgu Bliskiego}
\resizebox{\textwidth}{!}{
    \begin{tabular}{|l|l|l|l|l|l|l|l|l|l|l|l|l|l|l|l|l|l|l|l|l|l|l|l|l|l|}
    \hline
        Zasięg & 1 & 2 & 3 & 4 & 5 & 6 & 7 & 8 & 9 & 10 & 11 & 12 & 13 & 14 & 15 & 16 & 17 & 18 & 19 & 20 & 21 & 22 & 23 \\ \hline
        Koszt & -11 & -10 & -9 & -8 & -7 & -6 & -5 & -4 & -3 & -2 & -1 & 0 & 1 & 2 & 3 & 4 & 5 & 6 & 7 & 8 & 9 & 10 & 11 \\ \hline
    \end{tabular}
    }
\end{table}

\begin{table}[h]
\caption{Kosz zmiany Zasięgu Dalekiego}
    \begin{tabular}{|l|l|l|l|l|l|l|l|l|l|l|l|l|l|l|l|l|l|l|l|l|l|l|l|l|l|l|l|l|l|l|}
    \hline
    Zasięg & 1 & 2 & 3 & 4 & 5 & 6 & 7 & 8 & 9 & 10 & 11 & 12 & 13 & 14 & 15 & 16  \\ \hline
    Koszt & -15 & -14 & -13 & -12 & -11 & -10 & -9 & -8 & -7 & -6 & -5 & -4 & -3 & -2 & -1 & 0 \\ \hline
    Zasięg & 17 & 18 & 19 & 20 & 21 & 22 & 23 & 24 & 25 & 26 & 27 & 28 & 29 & 30 & 31 & 32 \\ \hline
    Koszt & 1 & 2 & 3 & 4 & 5 & 6 & 7 & 8 & 9 & 10 & 11 & 12 & 13 & 14 & 15 & 16 \\ \hline
    
    \end{tabular}
\end{table}


\myssection{Koszt Cechy Broni}

Kosz cechy broni wybieralny jest arbitralnie przez autora. 

\section{Umiejętności specjalne}

\section{Wyposażenie}

\section{Czary}